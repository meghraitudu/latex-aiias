\section{Dear Stacey,}
Thank you for reaching out to me and for your question. It is true that I am right now studying anew the doctrines of the Seventh-day Adventist Church and it is a great privilege to share with you what I have learned.
I am part of the Adventist church myself and happily so. You need to know that every doctrine in itself has the goal to lead you not just to some mental understanding but they actually are about a person. This person is Jesus. It is my desire that through reading about the doctrines in the way I will describe them you can see Him through them. I don't know how much you have heard about Jesus until now but let me just tell you, He changed my life for the better and I know He wants to do the same for you. Let's dive into what I believe can turn your world around. Of course this can only be a summary of in whole 28 beliefs and I hope that it starts you on a journey where you will go deeper into each subject for your own. There is always more to find out and more jewels to unpack.

\section{The Doctrine of God}
The first foundational topic is the one about God. As the Seventh-day Adventist church is a Christian denomination we believe there is one God in whom is the origin of all life. There are "many theories attempting to explain God, and the many arguments for and against His existence, show that human wisdom cannot penetrate the divine" \citep[p. 17]{ministerial1988seventh}. Although God can not be fully understood He wants us to get to know Him. How do we know this? There is evidence all around us. We look outside into the blue sky, we see flowers and animals of so many different kinds and we study the human body. When we look into the eyes of a newborn baby and we marvel about the miracle of life. King David wrote in the Psalms: "The heavens declare the glory of God; and the firmament shows His handiwork" (Psalm 19:1). Just by looking into nature we are able to learn a lot about God. However there are natural disasters, there are animals suffering, there is destruction. Nature alone would not give us an accurate picture of God. That is why He gave us a special revelation - the Bible. This book is the foundation for all the doctrines of the Adventist church. Every topic you will read about in this paper is based on the study of Scripture.\\
The Bible is a story about God in search of mankind. In this sense Adventism and "Christianity is not a record of a man's quest for God; it is the product of God's revelation of Himself and His purposes to man" \citep[p. 15]{lewis1978decide}. Through this book God is not only communicating to the world but He is talking to each human being individually. In the Bible we read of people like you and me and their relationship to God. I personally have experienced how studying the Word of God has spoken directly to my heart. When we let the general revelation of nature and the special revelation of Scripture lead us we can have a personal revelation of this God in our life.\\

\section{The Doctrine of Man}
So far we have learned that God wants us to know Him. Why is God interested in us? Genesis, the first book of the Bible, reveals that He created us. The Bible starts with these words: "In the beginning God created the heavens and the earth" (Genesis 1:1). God is omnipotent, He can create something out of nothing. In six days He made our world. On the sixth day He made mankind. He took extra care for the creation of man. With everything else He spoke and it was. However when He created man he formed him from the dust of the earth and the woman He built out of the rip of the man. Why did God create in the first place? "Love motivates all that God does, for He is love" (1 John 4:8) \citep[p. 72]{ministerial1988seventh}. This communicates a very special message: You are a created being and loved by your creator. This gives you special worth and dignity. The fact that we are made and God isn't, shows why we will never fully understand Him. He is eternal and self-existing (John 5:26), He has all power and is sovereign (Psalm 115:3). Still He wants to reveal Himself to you because He loves you. This is one of the most precious truths we can understand.\\
In the creation account of mankind we learn something about the nature of how God created us. "Then the Lord God formed a man from the dust of the ground and breathed into his nostrils the breath of life, and the man became a living being" (Genesis 2:7). There are two elements involved. First there is the dust of the ground and then there is the breath of life. Out of these two elements God made man a living being. Nowhere does it say that man received a soul but he became one. This shows that if the breath of life is taken away there is nothing left of us. No soul that lives on forever. \citep{ministerial1988seventh}.
Another important aspect we learn is that God created man in His image (Genesis 1:26). Mankind was the summit of creation. Everything God created was given into our care. One author puts it this way: "It was his duty and privilege to make all the nature and all created beings that were placed under his rule, subservient to his will and purpose, in order that he and his whole glorious dominion might magnify the almighty Creator and Lord of the universe, Gen. 1:28; Ps. 8:4-9" \citep[p. 183]{berkhof1996systematic}. We as human beings got a responsibility to care for this planet. By doing this we should give glory to God. However something happened that changed everything.

\section{The Doctrine of Salvation}
At the end of each day of creation God said: "It is good" (Genesis 1). However when we look around and when we look inside of us we realize that this is not the case anymore. Unfortunately in the third chapter of the Bible we read that there was a separation between God, the source of life and all good things, and us humans. To understand what happened we need to know that there is an adversary called Satan. In the Bible he is called the deceiver and murderer (John 8:44). Once he had been an angel near to God's throne (Ezekiel 28:14). In Isaiah 14:12-14 we read that something changed in the heart of this angel. He wanted to be like God and pride changed Lucifer, a perfect angel into  Satan, the enemy of God. Satan didn't keep his thoughts to himself. He started a rebellion in heaven. God's character was misrepresented and a great controversy broke out in heaven. This is vital to understand because it shows that it is not God who brings suffering into our world but the enemy of God who wants us to believe that God is unjust, cruel and selfish. The sad part is that humankind believed him and mistrusted God. This led to sin, the transgression of the law of God, which ultimately means separation from our creator. In Genesis 3 you can read how it happened in detail. Through this decision that Eve made on that fatal day we all experience death and suffering. However we are not innocent. "People are ethically sinful; and when God counts their trespasses against them, he must view them as sinners, as enemies, as the objects of the divine wrath" \citep[p. 495]{ladd1993theology}. When we look into our life we realize that we all have made mistakes that hurt somebody or ourselves. That is sin and the wages of sin is death (Romans 6:23). This sounds like a hopeless situation.\\
The following lines are the revelation of God's love in a way that we will never fully comprehend it. God did not leave us in this situation alone. We deserve death but God longs to save us and that is why He came up with a plan before we even existed (1 Peter 1:19-21). One of the probably most popular verses in the Bible gets to the heart of it: "For God so loved the world that He gave His only begotten Son, that whoever believes in Him should not perish but have everlasting life" (John 3:16). What does this mean? 
The consequence of sin is death. In order that we would be able to live we needed an atonement. The SDA Bible Dictionary says about atonement: "Understood in terms of its original meaning, 'atonement' properly denotes a state of reconciliation that terminated a state of estrangement" \citep[p. 97]{neufeld1979seventh}. This atonement was made possible through Jesus.\\
We believe the Bible teaches that God is one Godhead existing out of three individual persons: God Father, God Son and God Holy Spirit. "The members of the Godhead are allied in the work of bringing people back into a union with their Creator" \citep[p. 108]{ministerial1988seventh}. As John 3:16 says, God Father sent His son Jesus to save us. How did He do it? He came to this earth as a human being. He lived a perfect life and then died a shameful death on a cross bearing the sins of everyone. However, because He was guiltless Jesus rose again after three days and through this, we can know that He is able to save all humans if they believe.\\
This is a very short summary of the plan of salvation and why it is needed. Yet it is key to understand the importance it has for your personal life. Salvation is a gift from God. The crucial question now is, how can we accept it? The Bible tells us that the first step is to repent from our sins and confess them to God. 1 John 1:9 promises us: "If we confess our sins, he is faithful and just and will forgive us our sins and purify us from all unrighteousness." God takes away the guilt, this is called justification. "Those who accept by faith that God has reconciled the world to Himself in Christ and who submit to Him will receive from God the invaluable gift of justification with its immediate fruit of peace with God" \citep[p. 32]{LaRondelle1980Salvation} When God looks at us He no longer sees our sins but He sees Jesus' perfect life. From now on we live in gratitude to our savior and experience the life long process of sanctification which means being changed by God from inside out and we will reflect His character more and more.

\section{The Doctrine of the Church}
"In every age there were witnesses for God" \citep[p. 61]{white1911GC}. Today we call those people church. The Seventh-day Adventist Church believes itself to be a movement called by God to be His witness in a very special time. This church is an official organization with members but this visible church is not the only church God has. Jesus himself said to the Jews, who once were the chosen people, the "church" of God, that he has other sheep in other folds (John 10:16). Whether being a member of a certain church or being not a member of a church is determining if we are part of the invisible church of God. The question is does Jesus know us. After Jesus resurrected He ascended back into heaven but promised His disciples to come back again to take us all home (Acts 1:11). When He will come back there will be people who said they knew Jesus but the Son of God will say I know you not (Matthew 7:21-23). What a dreadful thing to hear from Him. However we need not be afraid because as we learned earlier. Jesus wants to save us. What we need to be part of His church is a relationship with Him and not an outward sign that we belong to a denomination. That said, the visible church still has a major role to play because God himself gave it a great task.\\
God gave the church a special message for this time. This message is found in Revelation 14 and is called the Three Angel's Message (Revelation 14:6-12). Essentially it is the final call to every human being to worship the true God, to separate themselves from evil and to make people aware that the end of time is coming. Jesus' second coming is at hand and with that the final judgment of good and evil. This is good news because it sets an end to death and suffering. This message also gives the major signs of the people of God \citep{ministerial1988seventh}.
\begin{enumerate}
\item The faith of Jesus.
\item The commandments of God.
\item The testimony of Jesus (Revelation 19:10).
\end{enumerate}
We will come back to those signs in more detail later. Right now we want to look at one more sign that Jesus mentioned to His followers here on earth. He said: "By this all will know that you are My disciples, if you have love for one another" (John 13:35). The church should be a place "where people are loved, respected, and recognized as somebody. A place where people acknowledge that they need each other. Where talents are developed. Where people grow. Where everybody is fulfilled" \citep[p. 15]{Bradford1986church}. Jesus himself prayed for unity for the believers so they would be able to experience even the unity that He and His Father are having (John 17:20-23). This is remarkable and a topic that needs deeper thought and study. However to be part of the movement has a blessing in store as we can see in the prayer of Jesus.
\section{The Doctrine of the Christian Life}
Let us come back to the signs of the people of God in Revelation 14. What does it mean to live as a true believer? This is essentially what those signs tell us. Having the faith of Jesus means on the one hand believe and live as Jesus did and on the other hand believing in Jesus and telling others about it. That includes so many aspects. A few will be mentioned here. The first one is that Jesus lived here on earth as a human being. He became our living example. Jesus for instance got baptized by John the Baptist (Matthew 3:13-17). Baptism is a symbol of washing away the sin and starting a new life. It is an outward sign of what happens inside. "Through baptism the new Christian enters into the church and the covenant relationship with Jesus" \citep[p. 21]{Rice1986baptism}. Although Jesus never sinned he was baptized as our role model. He also told his disciples: "Go therefore and make disciples of all the nations, baptizing them in the name of the Father and of the Son and of the Holy Spirit" (Matthew 28:19). Disciples are followers of Jesus and are stewards of everything they have been given by God. Jesus showed that stewardship is "man's responsibility for, and use of, everything entrusted to him by God — life, physical being, time, talents and abilities, material possessions, opportunities to be of service to others, and his knowledge of truth" \citep[p. 1425]{neufeld1976seventh}. That is a high calling but we have the promise that "he who began a good work in you will bring it to completion" (Philippians 1:6). Jesus is our perfect helper.\\
Let us look at the second sign: keeping the commandments of God. Many believe that Jesus took away the Ten Commandments. However "we must not imagine that the coming of Christ has freed us from the authority of the law; for it is the eternal rule of a devout and holy life, and must, therefore, be as unchangeable as the justice of God" \citep[p. 277]{calvin1949john}. With that the Seventh-day Adventist Church does also hold up the Sabbath given at creation and commanded to keep holy in the fourth commandment. As the name of the church says this is Saturday the seventh day of the week. The Sabbath is a gift from God for mankind. God rested on this day, He blessed and sanctified it. He meant for us to rest, spend time with Him and others and remember that God created not only the world but He also creates a new heart in us. The third sign will be discussed in the next chapter.
\section{The Doctrine of last things}
At the end of Revelation 19:10 the sign of "having the testimony of Jesus" is given and also explained that it is the spirit of prophecy. Prophets are people that get a special message from God for His people. Often times they also have visions and God reveals to them what will happen in the future. Through the prophecies of the Old Testament and in Revelation we are able to see that we are living in the final years before Jesus will come back to take us home. As we already mentioned in this time God raised up a group of people to give to the world a special message (the Three Angel's Messages). Revelation 19:10 tells us now that God will also give the gift of prophecy to the movement. The Seventh-day Adventist Church believes that this happened in the person of Ellen G. White. She wrote many books in which she communicated everything God showed her. Major themes she talked about were Salvation, the character of God, the great controversy and health and this is by far not a complete list. She always reaffirmed and pointed back to the Bible as the basis of our beliefs. Her writings are a special gift from God as you can see for yourself, when reading it. I would recommend especially the book "Steps to Christ" as it is not long but full of inspiring insights into the relationship that God wants to have with us.\\
Ellen White was also shown how it will be when Jesus is coming back and with this topic I want to close. Jesus promised to come back to earth. He said He will put an end to death and suffering and make all things new. God will again dwell in our midst. Read the last two chapters of the Bible (Revelation 21-22). They show where we are going and where our goal is when we trust in Him. This hope helps through the challenging times we are experiencing right now.
\section{Conclusion}
I hope you can see how the doctrine of the Seventh-day Adventist Church is a doctrine about a person: Jesus. He knows you by name (Isaiah 43:1). Maybe you might now think: Those are all interesting things to know but how do I actually get to know Him personally? That is the essential question.
Jesus promised to be with us always (Matthew 28:20). We can communicate with Him at any time. Talking to Jesus is called praying. We can tell Him everything and He will answer in His way and time. He promised to send the Holy Spirit, the third person of the Godhead, and "he will guide you into all the truth" (John 16:13). Jesus is just waiting for you to ask Him to come into your life and fill it with hope, love and peace.
It is my prayer that you might be enriched by what you learned and you will continue your journey in getting to know your savior. I know He loves you and me and longs to be with us forever.